\documentclass{beamer}

\usepackage[utf8]{inputenc}
\usepackage{default}

\mode<presentation>
%{ \usetheme{boxes} }


\usetheme{Madrid}

\usepackage{times}
\usepackage{graphicx}
\usepackage{tabulary}
\usepackage{listings}
\usepackage{verbatimbox}
\usepackage{graphicx}
\usepackage{lmodern}
\usepackage[absolute,overlay]{textpos}
\usepackage{pgfpages}
\usepackage{color}
\usepackage{multicol}

\pgfdeclareimage[height=1.0cm]{logo_rcc}{../icons/logo_rcc.png}
\setlength{\TPHorizModule}{1mm}
\setlength{\TPVertModule}{1mm}
\newcommand{\RCCLogo}{
\begin{textblock}{14}(1.5,1.5)
  \pgfuseimage{logo_rcc}
\end{textblock}
}



\definecolor{mycolorcli}{RGB}{53,154,26}
\definecolor{mycolorcode}{RGB}{0,0,255}
\definecolor{mycolordef}{RGB}{255,0,0}
\definecolor{mycolorlink}{RGB}{184,4,255}


\title{\huge{How to study this tutorial}}
\author{Igor Yakushin \\ \texttt{ivy2@uchicago.edu}}
% \date{March 31, 2018}
\date{}

\definecolor{ChicagoMaroon}{RGB}{128,0,0}

\setbeamercolor{title}{bg=ChicagoMaroon}

\begin{document}

\setbeamertemplate{navigation symbols}{}

\setbeamercolor{fcolor}{fg=white,bg=ChicagoMaroon}
\setbeamertemplate{footline}{
\begin{beamercolorbox}[ht=4ex,leftskip=1.4cm,rightskip=.3cm]{fcolor}
\hrule
\vspace{0.1cm}
   \hfill \insertshortdate \hfill \insertframenumber/\inserttotalframenumber
\end{beamercolorbox}
}

\setbeamercolor{frametitle}{bg=ChicagoMaroon,fg=white}

\begin{frame}
%\RCCLogo
\titlepage
\end{frame}

\begin{frame}[fragile]
  \frametitle{How to study this tutorial}
  \begin{itemize}
  \item The main point of this tutorial is to show what useful tools are availabe in the Hadoop zoo,
    how to get started with various tools, how to use them on RCC
    clusters. Very often, getting started is the most difficult thing.
    
  \item The tutorial does not go deep into each tool. Once you know how to use those tools for simplest tasks, it would be much easier
    to understand the original documentation and go deeper on your own.
  \item Some tools, like Spark and Hive, are studied in more depth in Big
    Data Platform class
  \item Some tools, like JavaAPI, HBase are not usually studied in Big Data Platform class at all
  \item In the first pass, study all the material in the exact order they are presented without skipping anything
  \item Do yourself all the examples presented in the tutorial
  \end{itemize}
\end{frame}


\begin{frame}[fragile]
  \frametitle{How to study this tutorial}
  \begin{itemize}
  \item The videos do not repeat slides:
    they only show things that are easier to understand from videos.
  \item Therefore, do not skip slides
  \item After the first pass over the tutorial, you can use it as a reference and review topics in any order
    since it is organized in a modular way and it is easy to find what you might be looking for.    
  \item If something is still not clear, ask me questions by e-mail or come see me during my office hours
  \item To get all the examples so that you can run them yourself,
    rather than downloading them one by one from canvas, do on midway or Hadoop cluster:
    {\color{mycolorcli}
\begin{verbatim}
  git clone git@github.com:igory1999/Hadoop_video.git
\end{verbatim}
    }
  \end{itemize}
\end{frame}

\end{document}


