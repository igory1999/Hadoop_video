\documentclass{beamer}

\usepackage[utf8]{inputenc}
\usepackage{default}

\mode<presentation>
%{ \usetheme{boxes} }


\usetheme{Madrid}

\usepackage{times}
\usepackage{graphicx}
\usepackage{tabulary}
\usepackage{listings}
\usepackage{verbatimbox}
\usepackage{graphicx}
\usepackage{lmodern}
\usepackage[absolute,overlay]{textpos}
\usepackage{pgfpages}
\usepackage{color}
\usepackage{multicol}


\pgfdeclareimage[height=1.0cm]{spark}{../../icons/spark.png}
\newcommand{\SPARK}{
\begin{textblock}{14}(107.5,1.5)
  \pgfuseimage{spark}
\end{textblock}
}



\definecolor{mycolorcli}{RGB}{53,154,26}
\definecolor{mycolorcode}{RGB}{0,0,255}
\definecolor{mycolordef}{RGB}{255,0,0}
\definecolor{mycolorlink}{RGB}{184,4,255}

\setcounter{tocdepth}{3}

\title{\huge{RDD}}
\author{Igor Yakushin \\ \texttt{ivy2@uchicago.edu}}

\definecolor{ChicagoMaroon}{RGB}{128,0,0}

\setbeamercolor{title}{bg=ChicagoMaroon}

\begin{document}

\setbeamertemplate{navigation symbols}{}

\setbeamercolor{fcolor}{fg=white,bg=ChicagoMaroon}
\setbeamertemplate{footline}{
\begin{beamercolorbox}[ht=4ex,leftskip=1.4cm,rightskip=.3cm]{fcolor}
\hrule
\vspace{0.1cm}
   \hfill \insertshortdate \hfill \insertframenumber/\inserttotalframenumber
\end{beamercolorbox}
}

\setbeamercolor{frametitle}{bg=ChicagoMaroon,fg=white}

\begin{frame}
\SPARK
\titlepage
\end{frame}

\section{Introduction}
\begin{frame}[fragile]
  \frametitle{RDD}
  
  \begin{itemize}
  \item RDD - Resilient Distributed Dataset
  \item RDD - collection of records partitioned across the nodes and can be operated on in parallel
  \item RDD can be created from files in various supported formats or by transforming other RDDs
  \item One can ask Spark to {\color{mycolordef}cache RDD in memory or disk} for fast reuse
  \item RDDs automatically recover from node failure
  \item RDD supports two types of operations:
    \begin{itemize}
    \item {\color{mycolordef}Transformations} - create a new dataset from an existing one.
      Lazy evaluation - evaluated only when required by action.
    \item {\color{mycolordef}Actions} - return a value to the driver program after running a computation on the dataset.
    \end{itemize}
  \end{itemize}  

\end{frame}

\section{Transformations}
\begin{frame}
  \frametitle{RDD: transformations}
  Examples of RDD transformations:
  \begin{itemize}
  \item {\color{mycolorcode}map(func)} - transform each element by a applying a function
  \item {\color{mycolorcode}filter(func)} - select records satisfying boolean function
  \item {\color{mycolorcode}sample(withReplacement, fraction, seed)} - Sample a fraction fraction of the data, with or without replacement, using a given random number generator seed.
  \item {\color{mycolorcode}union(otherDataset), intersection(otherDataset)}
  \item {\color{mycolorcode}distinct([numTasks]))}
  \item {\color{mycolorcode}groupByKey([numTasks])}
  \item {\color{mycolorcode}sortByKey([ascending], [numTasks])}
  \item {\color{mycolorcode}pipe(command, [envVars])}
  \end{itemize}
\end{frame}

\section{Actions}
\begin{frame}[fragile]
  \frametitle{RDD: actions}
  Examples of RDD actions:
  \begin{itemize}
  \item {\color{mycolorcode}reduce(func)} - Aggregate the elements of the dataset using a function func
    (which takes two arguments and returns one). 
    The function should be commutative and associative so that it can be computed correctly in parallel.
  \item {\color{mycolorcode}collect()} - Return all the elements of the dataset as an array at the driver program.
  \item {\color{mycolorcode}count()}
  \item {\color{mycolorcode}take(n)} - Return first n elements of the results
  \item {\color{mycolorcode}countByKey()} - Only available on RDDs of type (K, V). Returns a hashmap of (K, Int) pairs with the count of each key.
  \item {\color{mycolorcode}foreach(func)} - Run a function func on each element of the dataset. 
    This is usually done for side effects such as updating an accumulator variable (see below) or interacting with external storage systems.
  \end{itemize}
\end{frame}

\end{document}
